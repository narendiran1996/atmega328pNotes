\documentclass{article}

\usepackage[a4paper, bottom=0.5in, top=0.5in, left=0.5in, right=0.5in]{geometry}
\usepackage{wrapfig}
\usepackage{natbib}
\usepackage{url}
\usepackage{xcolor}
\usepackage{caption}
\usepackage{hyperref}
\hypersetup{
    colorlinks=true,    
    urlcolor=cyan,
}

\usepackage{multirow}
\usepackage{bytefield}

\usepackage{amsfonts}
\usepackage{float}
\usepackage{enumitem}

\usepackage{tikz-timing}[2014/10/29]
\usetikztiminglibrary[rising arrows]{clockarrows}

\usepackage{minted}

\usepackage{xparse} % NewDocumentCommand, IfValueTF, IFBooleanTF
\usepackage{tikz-timing}[2014/10/29]
\NewDocumentCommand{\busref}{som}{\texttt{%
		#3%
		\IfValueTF{#2}{[#2]}{}%
		\IfBooleanTF{#1}{\#}{}%
}}


\newcommand{\bitFormat}[1]{\emph{\textbf{\textcolor{cyan}{#1}}}}

\newcommand{\regFormat}[1]{\textbf{\textcolor{magenta}{#1}}}

\newcommand{\pinFormat}[1]{\emph{\textcolor{red}{#1}}}


\usepackage{graphicx}
\graphicspath{ {./Resources/pics/} }



\title{ATmega328P - Miscellaneous}
\author{Narendiran S}
\date{\today}

\begin{document}
\maketitle


\textbf{ \LARGE "1" - Unprogrammed | "0" - Programmed}

\section{Fuse Bits}


\begin{itemize}
    \item Atmeta328P Has three fuse Byte.
\end{itemize}

\subsection{Extended Fuse Byte}
\vspace*{0.5cm}
\begin{bytefield}[bitformatting={\large\bfseries},
    endianness=big,bitwidth=0.125\linewidth]{8}
    \bitheader[lsb=0]{0-7} \\
    \bitbox{1}{\small -}
    \bitbox{1}{\small -}
    \bitbox{1}{\small -}
    \bitbox{1}{\small -}
    \bitbox{1}{\small -}
    \bitbox{1}{\small BODLEVEL2}
    \bitbox{1}{\small BODLEVEL1}
    \bitbox{1}{\small BODLEVEL0}\\
\end{bytefield}
\begin{itemize}
    \item $V_{BOT}$ - Brown-out Threshold Voltage
\end{itemize}

\begin{table}[H]
    \begin{center}
        \begin{tabular}{|c|c|c|c|}
            \hline
            \bitFormat{BODLEVEL[2:0]} & \textbf{Min $V_{BOT}$} & \textbf{Typ $V_{BOT}$} & \textbf{Max $V_{BOT}$}\\
            \hline
            111 & \multicolumn{3}{|c|}{BOD disabled}\\
            \hline
            101 & 2.5 & 2.7 & 2.9\\
            \hline
            100 & 4.0 & 4.3 & 4.6\\
            \hline
        \end{tabular}
    \end{center}
\end{table}

\subsection{High Fuse Byte}
\vspace*{0.5cm}
\begin{bytefield}[bitformatting={\large\bfseries},
    endianness=big,bitwidth=0.125\linewidth]{8}
    \bitheader[lsb=0]{0-7} \\
    \bitbox{1}{\small RSTDISBL}
    \bitbox{1}{\small DWEN}
    \bitbox{1}{\small SPIEN}
    \bitbox{1}{\small WDTON}
    \bitbox{1}{\small EESAVE}
    \bitbox{1}{\small BOOTSZ1}
    \bitbox{1}{\small BOOTSZ0}
    \bitbox{1}{\small BOOTRST}\\
\end{bytefield}

\begin{table}[H]
    \begin{center}
        \begin{tabular}{|c|p{8cm}|c|}
            \hline
            \textbf{High Fuse Byte} & \textbf{Description} & \textbf{Default Value}\\
            \hline
            RSTDISBL & External reset disable - disable external reset pin and use as input or output pin & 1\\
            \hline
            DWEN & debugWIRE enable - enabled debugWIRE interface & 1\\
            \hline
            SPIEN & Enable serial program and data downloading & 0 - enable SPI programming\\
            \hline
            WDTON & Watchdog timer always On & 1\\
            \hline
            EESAVE & EEPROM memory is preserved through chip erase & 1\\
            \hline
            BOOTSZ1 & Select boot size & 0\\
            \hline
            BOOTSZ0 & Select boot size & 0\\
            \hline
            BOOTRST & Select reset vector - select reset vector location & 1\\            
            \hline
        \end{tabular}
    \end{center}
\end{table}


\begin{table}[H]
    \begin{center}
        \begin{tabular}{|c|c|c|c|c|}
            \hline
            \textbf{BOOTSZ[1:0]} & \textbf{Boot size} & \textbf{Pages} & \textbf{Application Flash Section} & \textbf{Boot Loader Falash Section}\\
            \hline
            11 & 256 words & 4 & 0x0000 - 0x3EFF  & 0x3F00 - 0x3FFF\\
            10 & 512 words & 8 & 0x0000 - 0x3DFF  & 0x3E00 - 0x3FFF\\
            01 & 1024 words & 16 & 0x0000 - 0x3BFF  & 0x3C00 - 0x3FFF\\
            00 & 2048 words & 32 & 0x0000 - 0x37FF  & 0x3800 - 0x3FFF\\
            \hline
        \end{tabular}
    \end{center}
\end{table}

\subsection{Low Fuse Byte}
\vspace*{0.5cm}
\begin{bytefield}[bitformatting={\large\bfseries},
    endianness=big,bitwidth=0.125\linewidth]{8}
    \bitheader[lsb=0]{0-7} \\
    \bitbox{1}{\small CKDIV8}
    \bitbox{1}{\small CKOUT}
    \bitbox{1}{\small SUT1}
    \bitbox{1}{\small SUT0}
    \bitbox{1}{\small CKSEL3}
    \bitbox{1}{\small CKSEL2}
    \bitbox{1}{\small CKSEL1}
    \bitbox{1}{\small CKSEL0}\\
\end{bytefield}

\begin{table}[H]
    \begin{center}
        \begin{tabular}{|c|c|c|}
            \hline
            \textbf{High Fuse Byte} & \textbf{Description} & \textbf{Default Value}\\
            \hline
            CKDIV8 & Divide clock by 8 & 0 - Divide clock by 8 and use as CPU clock\\
            \hline
            CKOUT & Clock output & 1\\
            \hline
            SUT1 & Select start-up time & 1\\
            \hline
            SUT0 & Select start-up time & 0\\
            \hline
            CKSEL3 & Select clock source & 0\\
            \hline
            CKSEL2 & Select clock source & 0\\
            \hline
            CKSEL1 & Select clock source & 1\\
            \hline
            CKSEL0 & Select clock source& 1\\            
            \hline
        \end{tabular}
    \end{center}
\end{table}

\begin{table}[H]
    \begin{center}
        \begin{tabular}{c|c}
            \textbf{\bitFormat{CKSEL[3:0]}} & \textbf{Device Clocking Option}\\
            \hline
            1111 - 1000 & Low power crystall oscillator\\
            0111 - 0110 & Full swing crystal oscillator\\
            0101 - 0100 & Low frequency crystal oscillator\\
            0011 & Internal 128kHz RC oscillator\\
            0010 & Calibrated internal RC oscillator\\
            0000 & External clock            
        \end{tabular}
    \end{center}
\end{table}

\section{Signature Bytes}
\begin{itemize}
    \item All Atmel microcontrollers have a three-byte Signature code which identifies the devices.
\end{itemize}

\begin{table}[H]
    \begin{center}
        \begin{tabular}{|c|c|c|c|}
            \hline
            \multirow{2}{*}{\textbf{Part}} & \multicolumn{3}{c|}{\textbf{Sinature Bytes Address}}\\
   
            \cline{2-4}  &  \textbf{0x000} & \textbf{0x001} & \textbf{0x002}\\
            \hline
            ATmega328P & 0x1E & 0x95 & 0x0F\\
            \hline
        \end{tabular}
    \end{center}
\end{table}


\section{Calibration Byte}
\begin{itemize}
    \item The Atmel ATmega328P has a byte calibration value for the internal RC oscillator.
    \item This byte resides in the high byte of address 0x000 in the signature address space.
    \item During reset, this byte is automatically written into the \regFormat{OSCCAL} register to ensure correct frequency of the calibrated RC oscillator.
\end{itemize}
\end{document}
