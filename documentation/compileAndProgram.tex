% \documentclass{article}

% \usepackage{NeededPackages}
% \begin{document}

\section{GCC\texorpdfstring{\cite{toolChain}}{}}
\begin{itemize}
    \item GNU Compiler Collectin - compiler system.
    \item Supports various language, processor and host operating system.
    \item AVR GCC - Reffering to GCC targeting AVR.
    \item AVR GCC translates high-level langues to assembly.
    \item AVR GCC three language - C, C++, Ada.
\end{itemize}

\section{GNU Binutils\texorpdfstring{\cite{toolChain}}{}} 
Source : \href{https://www.nongnu.org/avr-libc/user-manual/overview.html}{Tool Chain overview}
\begin{itemize}
    \item Binary Utilites - contains the GNU assembler(gas), GNU linker(ld), etc.
    \item avr-as - The assembler.
    \item avr-ld - The linker.
    \item avr-objcopy - Copy and translate object files to different format.
    \item avr-objdump - Display information from object file including disassembly.
    \item avr-size - List section sizes and total sizes.
    \item avr-nm - List symbols from objects files.
    \item avr-strings - List printable strings from files.
    \item avr-readelf - Display contents of ELF files.
    \item avr-addr2line - Convert addresses to file and line.
\end{itemize}

\section{avr-lib\texorpdfstring{\cite{toolChain}}{}}
\quad Open source standard C Libary - standard C Libary and Libary function specifit to AVR.



\section{Compiler Options\texorpdfstring{\cite{compilerOptimize}}{}} \label{optimization}
\emph{-On} -- for optimization level; n indicates the optimization level; 0 being the default no optimization; 
\begin{itemize}
    \item \emph{-O0} -- reduces compilation time and this is default
    \item \emph{-O} and \emph{-O1} -- the compiler tries to reduce code size and execution time, without performing any optimizations that take a great deal of compilation time.
    \item \emph{-O2} --  Optimize even more than \emph{-O1}
    \item \emph{-O3} --  Optimize even more than \emph{-O2}
    \item \emph{-Os} -- Optimize for size and enables all \emph{-O2} optimization but not expected to increase code size
    \item \emph{-Ofast} -- enables all \emph{-O3} optimzations and disregarads strict standard compliance
    \item \emph{-Og} -- Optimize for debugging experience
\end{itemize}


\section{Compilation} 
\begin{itemize}
    \item Will create the .obj object binary files.
    \item Use \emph{avr-gcc} along with following options
    \begin{itemize}
        \item Optimization option \emph{-On} - use \emph{-Os} generally.
        \item Warning option \emph{-Wall} - enables all the warning
        \item Debug option \emph{-g} - Produce debugging information.
        \item MCU option \emph{-mmcu} - the actual MCU - \href{https://www.nongnu.org/avr-libc/user-manual/#supp_devices}{Supported MCU}
        \item C file option \emph{-c} - the actual c file.
        \item Output file name \emph{-o} - Output file name
    \end{itemize}
    \item To see the object binary file use \emph{avr-objdump -S fileName.o}
    \mint[bgcolor=lightgray]{shell}{avr-gcc -Os -Wall -g -mmcu=atmega8 -c hello.c -o hello.o}
\end{itemize}


\section{Linking}
\begin{itemize}
    \item Link the bianary object file to binary elf file.
    \item Use \emph{avr-gcc} along with following options
    \begin{itemize}
        \item Optimization option \emph{-On} - use \emph{-Os} generally.
        \item Warning option \emph{-Wall} - enables all the warning
        \item Debug option \emph{-g} - Produce debugging information.
        \item MCU option \emph{-mmcu} - the actual MCU - \href{https://www.nongnu.org/avr-libc/user-manual/#supp_devices}{Supported MCU}
        \item .obj file option - the actual .obj file.
        \item Output file name \emph{-o} - Output file name

    \end{itemize}
    \item To see the object binary file use \emph{avr-objdump -S fileName.elf}
    \mint[bgcolor=lightgray]{shell}{avr-gcc -Os -Wall -g -mmcu=atmega8 hello.o -o hello.elf}
\end{itemize}

\section{Generating the hex file}
\begin{itemize}
    \item The Intel hex file is what we program into procesosr.
    \item Use \emph{avr-objcopy} along with following options
    \begin{itemize}
        \item section Option \emph{-j} -- which sections to copy - generally .text and .data section
        \item Output format option \emph{-O} - what Output format should be used - eg) ihex
        \item The input .elf file
        \item The output .hex file
    \end{itemize}
    \mint[bgcolor=lightgray]{shell}{avr-objcopy -j .text -j .data -O ihex hello.elf hello.hex}

\end{itemize}

\section{AVRDUDE\texorpdfstring{\cite{avrdude}}{}}
\subsection{Introduction}
\begin{itemize}
    \item AVRDUDE - AVR Downloader UploDEr is a program for downloading and uploading the on-chip memories of Atmel's AVR microcontroller.
    \item Can program Flash, EEPROM, fuse ,lock bits and  signature bytes.
    \item Can read or write all chip memory types mentioned above.
    \item Supports varieous programmers from STK500, AVRISP, mkII, JTAG ICE, PPI, serial bit-bang adapters, etc.
    \item The STK500, JTAG ICE, etc uses serial port to communicate.
    \item The JTAGICE, AVRISP, USBasp, USBtinyISP uses USB using \emph{libusb}.
\end{itemize}


\subsection{Command Line Options}
\begin{itemize}
    \item \emph{-p partno} -- the mandatory option which specifies the MCU.
    \item \emph{-b baudrate} -- Specify the Baudrate.
    \item \emph{-c programmer-id} -- Specify the pgorammer used. eg)arduino, avrisp, avrisp2, avrispmkII, avrispv2, jtag1, stk500, stk500v1, stk500v2, usbasp, usbtiny, etc.
    \item \emph{-C config-file} -- Configuration data file.
    \item \emph{-e} -- Causes a chip erase of FLash ROM, EEPROM to 0xff and clears all lock bits.
    \item \emph{-F} -- Override device signature check.
    \item \emph{-P port} -- Specifty the port to be used.
    \item \emph{-u} -- Used if you want to write fuse bits - this cuases disabling the safemode for fuse bits.
    \item \emph{-t} -- uses interactive terminal mode instead of up or downloading files.
    \item \emph{-v} -- Verbose
    \item \emph{-U memtype:op:filename[:format]}
    \begin{itemize}
        \item \emph{memtype} -- Memory types are 
        \begin{enumerate}[label=(\roman*)]
            \item \emph{calibration} -- One or more bytes of RC oscillator calibration data.
            \item \emph{eeprom} -- The EEPROM.
            \item \emph{efuse} -- The extended fuse byte
            \item \emph{flash} -- The flash ROM of device
            \item \emph{fuse} -- The fuse byte in devices with a single fuse byte.
            \item \emph{hfuse} -- The high fuse byte.
            \item \emph{lfuse} -- The low fuse byte.
            \item \emph{lock} - The lock byte.
            \item \emph{signature} - The three device signature byte (device ID).
        \end{enumerate}
        \item \emph{op} -- Operations are
        \begin{enumerate}[label=(\roman*)]
            \item \emph{r} -- Read the specifed device memory and write to specified file.
            \item \emph{w} -- read the specifed file and write to specifed device memory.
            \item \emph{v} -- read the specified device memory and the specified file and perform a verify operation .
        \end{enumerate}
        \item The \emph{filename} can be either a fileName,  immediate byte value (in decimal, binary,hexadecimal, etc)
        \item \emph{format} is optional and can 
        \begin{enumerate}
            \item \emph{i} - Intel hex
            \item \emph{r} - raw binary
            \item \emph{e} - the elf files
            \item \emph{m} - immediate mode
            \item \emph{d} - decimal
            \item \emph{b} - binary(0b)
            \item \emph{d} - hexadecimal(0x)
        \end{enumerate}
    \end{itemize}
\end{itemize}


\subsection{Example}

\subsubsection{Downloading hex file into device Flash}
\begin{minted}[bgcolor=lightgray]{bash}
    avrdude -p atmega8 -b 19200 -c stk500 -p /dev/ttyUSB0 -v -U flash:w:hello.hex:i
\end{minted}

\subsubsection{Uploading Flash from device into file}
\begin{minted}[bgcolor=lightgray]{bash}
    avrdude -p atmega8 -b 19200 -c stk500 -p /dev/ttyUSB0 -v -U flash:r:"./readFashMemory.bin":r
\end{minted}

\subsubsection{Reading Device signature}
\begin{minted}[bgcolor=lightgray]{bash}
    avrdude -p atmega8 -b 19200 -c stk500 -p /dev/ttyUSB0 -v -U signature:r:"deviceSinature.text":h
\end{minted}


\subsection{Writing the High fuse}
\begin{minted}[bgcolor=lightgray]{bash}
    avrdude -p atmega8 -b 19200 -c stk500 -p /dev/ttyUSB0 -v -U hfuse:w:0x65:m
\end{minted}

Also see this\cite{alsosee}.

% \bibliographystyle{plain}
% \bibliography{ReferenceM}

% \end{document}